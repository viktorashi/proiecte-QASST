\documentclass{article}

% Language setting
% Replace `english' with e.g. `spanish' to change the document language
\usepackage[english]{babel}
\usepackage{float}

% Set page size and margins
% Replace `letterpaper' with `a4paper' for UK/EU standard size
\usepackage[letterpaper,top=2cm,bottom=2cm,left=3cm,right=3cm,marginparwidth=1.75cm]{geometry}

% Useful packages
\usepackage{amsmath}
\usepackage{graphicx}
\usepackage[colorlinks=true, allcolors=blue]{hyperref}
\usepackage{xcolor}

\title{\textbf{QASST Project}}
\author{
Stan Ioan-Victor, ioan.victor.stan@ubbcluj.ro\\
Sebastian-George Hojda, sebastian.hojda@ubbcluj.ro\\
Timotei Copaciu, timotei.copaciu@ubbcluj.ro
}

\newpage

\begin{document}
\maketitle


\tableofcontents

\newpage

\textcolor{red}{Please remove the text in blue/magenta from the sections.}

\section{Software Tested}
\label{label:Software_tested}

\href{https://github.com/Commando-X/vuln-bank}{Vuln-Bank} \cite{ref:vulnBankRepo} is the application we're testing today. I know you might be tired of all the intentionally vulnerable apps, but it'll be very unlikely for us to stay and try to find issues in actual apps we use, because we're still learning, so what can we do \dots

Vuln-Bank is just one of the few ones where the issues that you can exploit.



\section{Approach on Security}
\label{label:Approach}

We're tackling both defensive and offensive, Timotei will be testing the application from a defensive perspective, and Victor with Sebi will be doing it offensively.

All 3 of us are developers, working in the industry, in the beginning of our careers.

The main objective of this bug-hunt was so that we can learn more about triaging, exploiting and defending against bugs.

The tools we've used were: 
\begin{itemize}
    \item Snyk.
    etc.etc.
\end{itemize}


\section{Strategy Applied}
\label{label:Strategy}

\subsection{Offensive approach}
\textcolor{blue}{
\textit{Using the offensive approach:}\\
(1) Each student who addresses the offensive approach should describe shortly 2 vulnerabilities that will be investigated and remediated later.\\
(2) Include a table with the features and the corresponding investigated vulnerabilities.}

Vulnerability 01. several details (what, how, ...) \textit{CWE-89: SQL Injection} or category \textit{A03:2021-Injection}  \cite{Vuln001} is investigated as data... [reasoning about choosing to investigate the particular vulnerability.]


\subsection{Defensive approach}

\par The defensive strategy adopted for this project focuses on Static Application Security Testing (SAST) to identify vulnerabilities early in the development lifecycle. Our primary objective was to assess the code quality and conformance to security specifications without executing the application. To achieve this, we employed Snyk, a specialized SAST tool, to perform a comprehensive scan of the source code. This approach allows us to detect "code smells" and insecure coding patterns that might otherwise be missed during manual reviews.

\par The investigation specifically targeted the critical components of the application responsible for data handling and user authentication. We configured the scanning strategy to focus on the backend database interface files and the HTTP session management modules, as these are high-risk areas for data leaks. By isolating these aspects, we aimed to uncover structural weaknesses related to input validation and session security configuration.

\par Our analysis strategy involved filtering the automated report generated by the tool to prioritize actionable threats. We focused on identifying vulnerabilities with High and Medium severity levels that directly impact the confidentiality and integrity of user data. Specifically, we looked for patterns indicating SQL Injection flaws in the query structures and Improper Certificate Validation in the communication layer. Additionally, we investigated the frontend code for Cross-site Scripting (XSS) risks and inspected the cookie generation logic for missing security attributes.

\par Finally, the applied strategy includes a remediation verification phase. After identifying and manually reproducing the reported vulnerabilities to rule out false positives, we plan to implement the code fixes suggested by the Snyk scanner. This "scan-fix-rescan" cycle ensures that the security attributes are effectively restored and that the remediation steps do not introduce regression issues.

\section{Vulnerabilities}
\label{label:Vulnerabilities}

\par \textbf{Vulnerability 01: SQL Injection (CWE-89 \cite{vul01:sqlInjection} - Severity: High.}
\par This vulnerability was identified by the Snyk SAST tool with a high priority score of 839. The issue resides in the \texttt{app.py} file, within the \texttt{api\_v1\_forgot\_password} function. The scanner detected that the application processes the \texttt{/api/v1/forgot-password} POST request by taking the \texttt{username} input directly from the JSON body and embedding it into a database query.
\par The specific code flaw is the use of a Python f-string to construct the SQL command: \texttt{f"SELECT id FROM users WHERE username='\{username\}'"}. Because the input is not sanitized or parameterized before being passed to \texttt{execute\_query}, an attacker can inject malicious SQL syntax to manipulate the query logic. 

\par \textbf{Vulnerability 02: Improper Certificate Validation (CWE-295 \cite{vul02:certValidation}) - Severity: High.}
\par This vulnerability was flagged with a priority score of 802 and is located in the \texttt{app.py} file, inside the \texttt{upload\_profile\_picture\_url} function. The scanner identified a critical security misconfiguration in how the application handles external HTTP requests. Specifically, the code attempts to fetch an image from a user-supplied URL using the Python \texttt{requests} library: \texttt{requests.get(..., verify=False)}.
\par By explicitly setting the \texttt{verify} parameter to \texttt{False}, the application disables SSL/TLS certificate verification. This effectively allows the application to trust any certificate presented by the server, including self-signed or malicious ones. This vulnerability opens the door to Man-in-the-Middle (MitM) attacks, where an attacker could intercept the connection, inspect the traffic, or inject malicious data into the response stream.

\par \textbf{Vulnerability 03: DOM-based Cross-site Scripting (CWE-79 \cite{vul03:crossSite}) - Severity: Medium.}
\par Snyk flagged this vulnerability with a priority score of 602. The security flaw is located in the frontend template file \texttt{templates/forgot\_password.html}. The issue occurs within the JavaScript code responsible for handling the "Forgot Password" form submission. After receiving a JSON response from the API, the application dynamically updates the user interface using the command: \texttt{document.getElementById('message').innerHTML = data.message}.
\par The use of \texttt{innerHTML} allows the browser to parse the data as HTML elements rather than plain text. If the backend response—specifically the \texttt{data.message} field—contains malicious JavaScript (e.g., \texttt{<script>alert(1)</script>}), it will be executed immediately in the user's browser. This creates a DOM-based XSS vulnerability.

\par \textbf{Vulnerability 04: Sensitive Cookie in HTTPS Session Without 'Secure' Attribute (CWE-614 \cite{vul04:sensCookie}) - Severity: Low.}
\par This vulnerability was identified by Snyk with a priority score of 402. The issue is located in the \texttt{app.py}, within the \texttt{login} function. Upon successful user authentication, the application generates a JWT session token and sets it in the user's browser using \texttt{response.set\_cookie('token', token, httponly=True)}.
\par While the developer correctly included the \texttt{httponly=True} flag to prevent client-side scripts from accessing the cookie, the \texttt{secure} flag was omitted. By default, this flag is set to \texttt{False}, meaning the sensitive session token can be transmitted over unencrypted HTTP connections. This configuration exposes the user's session to Man-in-the-Middle (MitM) attacks, where an attacker could intercept the plain-text cookie.





\section{Aimed Assets}
\label{label:Assets}
The effect these vulnerabilites have on the assets is presented below.
\begin{table}[H]
\centering
\caption{Vulnerability Asset Mapping}
\label{tab:vulnerability_assets}
\begin{tabular}{|l|l|p{6cm}|}
\hline
\textbf{Vuln. number} & \textbf{vulnerability name} & \textbf{affected assets} \\ \hline
1 & SQL injection & \textbf{file:} \texttt{app.py} \newline \textbf{function:} \texttt{api\_v1\_forgot\_password} \newline \textbf{data:} User database (IDs, credentials) \\ \hline
2 & Improper certificate validation & \textbf{file:} \texttt{app.py} \newline \textbf{function:} \texttt{upload\_profile\_picture\_url} \newline \textbf{communication:} External HTTP requests (SSL/TLS) \\ \hline
3 & DOM-based Cross-site scripting & \textbf{file:} \texttt{templates/forgot\_password.html} \newline \textbf{component:} DOM element \texttt{'message'} \newline \textbf{data:} User browser session, frontend UI \\ \hline
4 & Sensitive cookie without 'secure' attribute & \textbf{file:} \texttt{app.py} \newline \textbf{function:} \texttt{login} \newline \textbf{token:} JWT session token (\texttt{'token'} cookie) \\ \hline
\end{tabular}
\end{table}

\section{Affected Security Attributes}
\label{label:SecurityAttributes}

\textcolor{blue}{
Provide arguments (1-2 paragraphs) on how the security attributes may be affected by the existence of the studied vulnerabilities.}

The SQL injection affects Confidentiality since everyone can just leak data; Integrity, since anyone can change it; Availability, causing DOS if the data is deleted; undermining Accountability, since any mallicious party can perform actions as if they were a legitimiate actor. \\

All 1,2, and 4 affect non-repudiation since, for 1. Database records can be maipulated, for 2 and 4, a man-in-the-middle can get ahold of a user's access tokens and perform actions on their behalf.

\begin{table}[h!]
\centering
\caption{Security Attributes Affected by Vulnerabilities}
\label{tab:vulnerability_ciaan}
\begin{tabular}{|l|l|p{4cm}|p{4cm}|}
\hline
\textbf{Vuln. number} & \textbf{Name} & \textbf{primary assets} & \textbf{Affected CIAAN attributes} \\ \hline
1 & SQL injection & user database, \texttt{app.py} & Confidentiality, Integrity, Availability, Accountability, Non-repudiation \\ \hline
2 & Improper Certificate Validation & Network Traffic, External Data & Confidentiality, Integrity \\ \hline
3 & DOM-based XSS & User Browser, \texttt{forgot\_password.html} & Confidentiality, Integrity, Non-repudiation \\ \hline
4 & Missing 'Secure' Cookie Attribute & JWT Session Token & Confidentiality, Non-repudiation \\ \hline
\end{tabular}
\end{table}


\section{Tools Employed}
\label{label:Tools}

\par Snyk Code was employed as the primary Static Application Security Testing (SAST) tool to implement the defensive security strategy. This tool integrates directly into the development workflow to analyze the source code for security vulnerabilities without requiring code execution. For this investigation, we utilized Snyk's capability to parse the application's dependency tree and source files to identify known vulnerabilities (CVEs) and code quality issues, such as SQL injection risks and improper certificate validation. One significant benefit of using Snyk is its real-time scanning capability, which provided immediate feedback on security flaws as the code was being reviewed. A second key benefit is its actionable remediation advice, which not only identifies the line of code causing the issue but often provides context-aware fix suggestions, significantly reducing the time required for the remediation phase

\par \textcolor{blue}{
A new paragraph for tools used in offensive approach will go here}


\section{Test Design. Test Execution. Test Report}
\label{label:Tests}

\subsection{Defensive Approach}

\par In the defensive approach, the "Test Execution" phase consisted of performing a Static Application Security Testing (SAST) scan using Snyk. The tool parsed the application source code to identify patterns matching known Common Weakness Enumerations (CWEs).

\par The table below summarizes the four vulnerabilities selected for investigation, detailing their severity, the specific application feature affected, and the security attributes (from the CIA triad) compromised.

\begin{table}[h!]
\centering
\begin{tabular}{|p{4cm}|l|p{4cm}|l|}
\hline
\textbf{Vulnerability Name} & \textbf{Severity} & \textbf{Affected Feature} & \textbf{Security Attribute} \\ \hline
SQL Injection (CWE-89) & High & API: Password Reset & Confidentiality, Integrity \\ \hline
Improper Certificate Validation (CWE-295) & High & API: Profile Upload  & Confidentiality \\ \hline
DOM-based XSS (CWE-79) & Medium & UI: Forgot Password  & Integrity \\ \hline
Insecure Cookie Attribute (CWE-614) & Low & Auth: Login Session & Confidentiality \\ \hline
\end{tabular}
\caption{Summary of Vulnerabilities Detected via SAST Scan}
\label{tab:vuln_summary}
\end{table}

\subsubsection{Test Report Evidence}
\par The following figures provide evidence of the test execution results generated by the Snyk tool. These reports detail the specific lines of code responsible for the vulnerabilities and the estimated impact.

\begin{figure}[H]
    \centering
    \includegraphics[width=\textwidth]{Figures/vul_01-sql-injection.png} % SQL Injection
    \caption{SQL Injection Detection}
    \label{fig:sqli_evidence}
\end{figure}
\begin{figure}[H]
    \centering
    \includegraphics[width=\textwidth]{Figures/vul_02-certificate-validation.png} % Cert Validation
    \caption{Improper Certificate Validation}
    \label{fig:cert_evidence}
\end{figure}

\begin{figure}[H]
    \centering
    \includegraphics[width=\textwidth]{Figures/vul_03-corss-site-scripting.png} % XSS
    \caption{DOM-based XSS Detection}
    \label{fig:xss_evidence}
\end{figure}
\begin{figure}[H]
    \centering
    \includegraphics[width=\textwidth]{Figures/vul_04-sensitive-cookie.png} % Cookie
    \caption{Insecure Cookie Attribute}
    \label{fig:cookie_evidence}
\end{figure}

\begin{table} [H]
\centering
\begin{tabular}{l|l|l|l|l}
Feature & TC ID & Input1 & Input2 & Expected Output \\ \hline
F001  &TC01 & 42 & 15 & 100\\
F001  &TC02 & 1 & -2 & 3\\
F002  &TC03 & 111 & 90 & -74
\end{tabular}
\caption{\label{tab:TCs1}TCs table.}
\end{table}

Table \ref{tab:TCs1} shows the TCs designed to evaluate the vulnerability AAA over F001 and F002.



 \textcolor{blue}{\textit{For the offensive approach} $->$\\
(1) For each vulnerability, \\
-- indicate/identify at least one test design technique employed to design test cases.\\
-- fill in a table with at least 3 test cases (TCs) that may expose that vulnerability. Include in the table the name or id of the application feature affected by the vulnerability. \\
(2) For each TC indicate the input, expected output, and the actual result.\\
(3) Manual testing or tool-based testing can be used (e.g., fuzzing).\\
(4) Include screenshots or tool-generated reports for the performed scan/testing.}


\section{Vulnerability Exploit}
\label{label:Exploit}

\textcolor{blue}{\textit{For the defensive approach} $->$\\
-- Imagine an attack scenario and an attempt to exploit the vulnerability.\\
-- For successful attempts provide proof of the compromised assets, e.g., screenshot, data extracted, etc.} \\




\textcolor{blue}{\textit{For the offensive approach} $->$\\
-- Provide steps executed manually or a script that allows to exploit of the vulnerability and compromise of the asset(s).\\
-- Include proof of the asset compromised, i.e., screenshot, data exposed/received, etc.}


\subsection{Manual Exploit}
\label{label:ManualExploit}

\textcolor{blue}{
Seems like more for offensive \dots \\
-- commands provided to extract data\\
-- screenshots
}

\subsection{Automated Exploit (POC)}
\label{label:AutomatedExploit}

\textcolor{blue}{
Offensive:\\
- a piece of code (Python, Java, etc.) that automates how the vulnerability is exploited, such that data is extracted/changed/updated/compromised.
}


\section{Remediation Steps}
\label{sec:Remediation}

\subsection{Defensive Approach Remediation}

\par Following the recommendations provided by the Snyk SAST tool, we implemented fixes for the four selected vulnerabilities. The goal was to neutralize the security risks by modifying the source code to adhere to secure coding standards.

\subsubsection{Remediation of Vulnerability 01: SQL Injection}
\par \textbf{Reasoning:} The Snyk report identified that the use of Python f-strings to construct SQL queries allowed user input to alter the query logic. To fix this, we adopted the \textit{Parameterized Query} pattern (Prepared Statements). This is the industry-standard solution because it forces the database engine to treat user input strictly as data, effectively neutralizing any injected SQL commands.

\par \textbf{Code Change (\texttt{app.py}):}
\begin{itemize}
    \item \textbf{Before (Vulnerable):}
    \begin{verbatim}
    # Vulnerable: Input is directly embedded in the query string
    user = execute_query(
        f"SELECT id FROM users WHERE username='{username}'"
    )
    \end{verbatim}
    
    \item \textbf{After (Fixed):}
    \begin{verbatim}
    # Fixed: Using parameterized query syntax
    query = "SELECT id FROM users WHERE username=?"
    user = execute_query(query, (username,))
    \end{verbatim}
\end{itemize}

\subsubsection{Remediation of Vulnerability 02: Improper Certificate Validation}
\par \textbf{Reasoning:} The application was explicitly configured to ignore SSL certificate errors (\texttt{verify=False}), exposing it to Man-in-the-Middle attacks. We remediated this by enforcing certificate validation. Although Snyk suggested simply removing the flag (since \texttt{True} is the default), we explicitly set it to \texttt{True} to make the security intent clear to future developers.

\par \textbf{Code Change (\texttt{app.py}):}
\begin{itemize}
    \item \textbf{Before (Vulnerable):}
    \begin{verbatim}
    resp = requests.get(image_url, timeout=10, 
                        allow_redirects=True, verify=False)
    \end{verbatim}
    
    \item \textbf{After (Fixed):}
    \begin{verbatim}
    resp = requests.get(image_url, timeout=10, 
                        allow_redirects=True, verify=True)
    \end{verbatim}
\end{itemize}

\subsubsection{Remediation of Vulnerability 03: Cross-site Scripting (XSS)}
\par \textbf{Reasoning:} The frontend code used the \texttt{innerHTML} property to display user-controlled messages. This allows the browser to render any HTML tags (including malicious \texttt{<script>} tags) contained in the message. We chose to replace this with the \texttt{textContent} property. This solution was selected because it is the most performant way to strip HTML context and render the output as pure text, fully preventing DOM-based XSS.

\par \textbf{Code Change (\texttt{templates/forgot\_password.html}):}
\begin{itemize}
    \item \textbf{Before (Vulnerable):}
    \begin{verbatim}
    document.getElementById('message').innerHTML = data.message;
    \end{verbatim}
    
    \item \textbf{After (Fixed):}
    \begin{verbatim}
    document.getElementById('message').textContent = data.message;
    \end{verbatim}
\end{itemize}

\subsubsection{Remediation of Vulnerability 04: Sensitive Cookie Without 'Secure' Attribute}
\par \textbf{Reasoning:} The session cookie was missing the \texttt{secure} flag, allowing it to be transmitted over unencrypted HTTP connections. We updated the cookie configuration to include \texttt{secure=True}. This ensures the browser will never send the cookie unless the connection is encrypted (HTTPS), protecting the session token from interception.

\par \textbf{Code Change (\texttt{app.py}):}
\begin{itemize}
    \item \textbf{Before (Vulnerable):}
    \begin{verbatim}
    response.set_cookie('token', token, httponly=True)
    \end{verbatim}
    
    \item \textbf{After (Fixed):}
    \begin{verbatim}
    response.set_cookie('token', token, httponly=True, secure=True)
    \end{verbatim}
\end{itemize}

\subsection{Verification (Re-Scan)}
\par After applying the remediation steps detailed above, we performed a secondary scan using Snyk to verify the removal of the vulnerabilities. The screenshots below illustrates the initial and updated report summary, confirming that the some issues have been resolved.

\begin{figure}[H]
\centering
\includegraphics[width=0.25\linewidth]{Figures/snyk-report-before-fix.png}
\caption{\label{fig:initial-report} Initial Snyk Scan Report Summary }
\end{figure}

\begin{figure}[H]
\centering
\includegraphics[width=0.25\linewidth]{Figures/snyk-report-before-fix.png}
\caption{\label{fig:initial-report} Snyk Re-Scan Report Summary showing reduced vulnerability count }
\end{figure}


\textcolor{blue}{\textit{For the offensive approach} $->$\\
(1) Indicate steps to provide vulnerability remediation. E.g., piece of code, particular recommendations, etc.}




\section{Vulnerability Reporting}
\label{label:Reporting}

\textcolor{blue}{\textbf{Each team member} should select a vulnerability from the detected ones and provide a short issue report that may refer to previous sections. Choose one of the vulnerability reporting strategies discussed during \textbf{Lecture 10} (\textit{week 11}), i.e., RIMSEC, CIA-TAR, STRIDE-R,  and describe the selected security issue. For each component of the acronym, include at least one sentence/idea referring to that aspect, e.g., for RIMSEC, the Reproducibility aspect is included in section xx, where…\\
In case no vulnerability was identified after the performed investigation, provide 2-3 paragraphs with reasoning on the best reporting strategy to be applied (from the 3 mentioned above or others), considering the security approach and strategy employed during the investigation.}

\section{Conclusions}
\label{label:Conclusions}

\textcolor{blue}{Include final conclusions, lessons learned and personal considerations while working on QASSTP (3-4 paragraphs).\\
You can focus on the following aspects: type of application to be tested, amount of knowledge to use (related or not to testing), tools required to apply, team collaboration, amount of time needed to fulfill the tasks, etc.}

\section{Other sections...}
\label{label:Other}

\textcolor{blue}{Please remove this section and all subsections.}
\subsection{How to include Figures}

First you have to upload the image file from your computer using the upload link in the file-tree menu. Then use the includegraphics command to include it in your document. Use the figure environment and the caption command to add a number and a caption to your figure. See the code for Figure \ref{fig:frog} in this section for an example.

Note that your figure will automatically be placed in the most appropriate place for it, given the surrounding text and taking into account other figures or tables that may be close by. You can find out more about adding images to your documents in this help article on \href{https://www.overleaf.com/learn/how-to/Including_images_on_Overleaf}{including images on Overleaf} \cite{ref:overleafIncludingImages}.

\bibliographystyle{alpha}
\bibliography{sample}

\end{document}
