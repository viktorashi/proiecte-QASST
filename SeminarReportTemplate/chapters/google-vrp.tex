The very idea of the Google bug bounty / Vulnerability Reward program (VRP) stemmed from the approach Donald Knuth, the creator of the typing system used by this very paper you're reading  \cite{GoogleBugHunters}  \\
Back in the 70s he just kept on going for (perhaps) way too long perfecting his book "The art of computer programming", so he thought maybe it'd be a better idea to let the people submit erratas for money. Thus, a primitive form of "bug bounty" program was born. \\
Maybe try to ship fast, and allow the people to review and correct your insecure code, rewarding them for it. \\
Now, the folks over at google established a formal set of criteria by which basement-dwellers have to abide.
\hrule
\subsection{Hacker's VRP Workflow \texorpdfstring{$\mathcal{W}_{\text{Hacker}}$}{W Hacker}}
The hacker's process is governed by adherence criteria to the program's rules:
\begin{enumerate}[label=\textbf{\arabic*}., align=left, leftmargin=0pt, itemsep=5pt]
    \item \textbf{Preparation and Discovery:}
    \begin{itemize}
        \item Review the official VRP \textbf{scope} and \textbf{rules} (e.g., the sstuff i'm boutta tell you).
        \item Actively search for a valid, security-relevant vulnerability ($V$) within an eligible Google asset, keeping in mind that they're starting to get a lot of AI slop garbage on their doorstep, so maybe something of use to them would be good.
    \end{itemize}
    \item \textbf{Report Submission ($\mathcal{R}$):}
    \begin{itemize}
        \item Submit a detailed, high-quality report $\mathcal{R}$ containing:
        \begin{itemize}
            \item Clear \textbf{Steps to Reproduce} (STR).
            \item A functional \textbf{Proof-of-Concept} (PoC) code or demonstration.
            \item Analysis of the \textbf{Security Impact} ($S$).
        \end{itemize}
      \item \textbf{Constraint:} The report must be submitted via the designated Google Bug Hunters platform. (duh)
    \end{itemize}
    \item \textbf{Coordinated Disclosure:}
    \begin{itemize}
        \item Maintain \textbf{confidentiality} of $V$ until Google has deployed a patch ($P$). Pre-patch public disclosure invalidates eligibility for a reward ($E \rightarrow 0$).
    \end{itemize}
    \item \textbf{Reward and Recognition:}
    \begin{itemize}
        \item If $V$ is valid, unique, and in-scope, receive a \textbf{Monetary Reward} ($M$) based on severity $S$.
        \item Potential for a \textbf{Bonus} ($B$) for exceptional reports (e.g., including a working fix or root cause analysis).
        \item Public \textbf{Hall of Fame} recognition ($H$).
    \end{itemize}
\end{enumerate}

\hrule

\subsection{Google's VRP Response (\texorpdfstring{$\mathcal{W}_{\text{Google}}$}{W Google}) }
Google's process centers on rapid triage, accurate severity assessment, remediation, and reward fulfillment.
\begin{enumerate}[label=\textbf{\arabic*}., align=left, leftmargin=0pt, itemsep=5pt]
    \item \textbf{Triage and Validation ($T$):}
    \begin{itemize}
        \item \textbf{Acknowledgement:} Initial confirmation of report reception.
        \item \textbf{Reproduction and Validation:} Security engineers attempt to replicate the bug using the provided PoC.
        \item \textbf{Criteria Check:} Determine if $V \in \text{Scope}$ and is not a duplicate ($\neg \text{Duplicate}$).
    \end{itemize}
    \item \textbf{Severity Assessment ($S$):}
    \begin{itemize}
        \item Assess the risk based on \textbf{Impact} (e.g., arbitrary code execution, sensitive data leakage) and \textbf{Exploitability} (complexity of exploitation).
        \item Assign a severity level ($\text{Critical, High, Medium, Low}$).
    \end{itemize}
    \item \textbf{Remediation and Fix ($P$):}
    \begin{itemize}
        \item File an internal bug ticket and assign it to the relevant Product Engineering Team.
        \item The team develops and deploys the corrective patch $P$ to mitigate $V$.
    \end{itemize}
    \item \textbf{Reward Determination ($M$):}
    \begin{itemize}
        \item An internal VRP panel reviews the validated report and assigned severity $S$.
        \item The final reward amount $M$ is calculated: $M = \text{BaseReward}(S) + B$.
        \item Payment is processed to the hacker.
    \end{itemize}
\end{enumerate}

\hrule

\subsection{The VRP Report Lifecycle}
The end-to-end process is modeled as a linear progression of stages, aiming for rapid Mean Time To Remediation (MTTR).
$$\mathcal{R}_{\text{Submit}} \xrightarrow{\text{Google}} T \xrightarrow{\text{Security Team}} S \xrightarrow{\text{Product Team}} P \xrightarrow{\text{VRP Panel}} M$$

\vspace{5pt}
\noindent Where:
\begin{itemize}
    \item $\mathcal{R}_{\text{Submit}}$: Hacker submits the report.
    \item $T$: Triage and Validation (some call that, google magic).
    \item $S$: Severity Assessment.
    \item $P$: Patch Development and Deployment (Fix).
    \item $M$: Reward Payment.
\end{itemize}


\subsection{An examplee noww}
There's an infamous hacker going around their bug bounty programs, that seems to be romanian, \cite{inspector-ambitious}, but he doesn't publish his reports, unfortunately, so I'll instead talk about \cite{NDevTK}, whose, out of all of the top spots, actually publishes their findings and replicability steps. \\

This exploit has to do with the way Google's old GitHub competitor was parsing their ... blabla si de aici nu prea mai stiu sincer vad eu dupa

