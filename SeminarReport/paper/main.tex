\documentclass{article}

% Language setting
% Replace `english' with e.g. `spanish' to change the document language
\usepackage[english]{babel}

% Set page size and margins
% Replace `letterpaper' with `a4paper' for UK/EU standard size
\usepackage[letterpaper,top=2cm,bottom=2cm,left=3cm,right=3cm,marginparwidth=1.75cm]{geometry}

% Useful packages
\usepackage{amsmath}
\usepackage{graphicx}
\usepackage[colorlinks=true, allcolors=blue]{hyperref}
\usepackage{xcolor}
\usepackage{amssymb}
\usepackage{fancyhdr}
\usepackage{enumitem}
\usepackage[utf8]{inputenc}
\usepackage{booktabs} % For better looking tables
\usepackage{caption} % For table caption control
\usepackage{listings}

\renewcommand{\headrulewidth}{0.4pt}
\renewcommand{\footrulewidth}{0pt}

\title{\textbf{SEMINAR REPORT TITLE}\\
Seminar Report}

\author{TEAM:\\
Stan Ioan-Victor, ioan.victor.stan@ubbcluj.ro\\
Sebastian-George Hojda, sebastian.hojda@ubbcluj.ro\\
Timotei Copaciu, timotei.copaciu@ubbcluj.ro
}


\pagestyle{fancy}

\date{\today}


\begin{document}

\maketitle

\tableofcontents

\newpage

\section{Introduction}

Software quality assurance has shifted from isolated internal testing to continuous, community-driven validation. In security, this change is embodied by Bug Bounty Programs (BBPs), which incentivize external researchers to find and report vulnerabilities \cite{wiki-bbp}. Traditional assurance tools such as SAST, DAST, and periodic penetration tests struggle with business logic flaws and contextual vulnerabilities that automated scanners cannot reliably detect \cite{bugcrowd-sdlc}. BBPs complement internal QA by introducing diverse human expertise capable of identifying race conditions, IDORs, and complex exploit chains often missed by automated tools.

This report examines BBPs as a modern QA mechanism, focusing on programs run by Uber, Meta, Google, and GitLab. It evaluates their structure, Safe Harbor policies, and operational metrics such as time-to-triage, payout models, and researcher engagement. The analysis shows that BBPs have evolved from ad-hoc initiatives into essential infrastructure supporting high-reliability software delivery.

\section{Relevance of Bug Bounties}

\subsection{Limitations of Traditional Models}

In the ISO/IEC 25010 framework, security is a core quality attribute. However, point-in-time penetration testing cannot keep pace with rapid CI/CD deployments \cite{miessler-pentest}. BBPs provide continuous, asynchronous testing in production environments, uncovering “unknown unknowns’’ that scanners routinely miss. Human researchers excel at detecting workflow abuses and logic flaws that fall outside signature-based detection \cite{securityjourney-compared}.

\subsection{Economic and Strategic Efficiency}

Unlike fixed-fee penetration tests, BBPs operate on a pay-for-results model: organizations pay only for valid findings \cite{deepstrike-vs}. Competitive rewards—such as Meta’s high-range payouts—encourage responsible disclosure over grey-market sales \cite{meta-bbp}. BBP data also reveals systemic weaknesses (e.g., recurring XSS or IDORs), enabling targeted developer training and architectural improvements.

\subsection{Community and Gamification}

Bug bounty platforms use leaderboards, bonuses, and reputation systems to sustain engagement. Programs such as Uber’s loyalty bonus system encourage deeper research rather than superficial scanning \cite{uber-welcome}. This creates an incentive structure aligned with real-world risk rather than checklist compliance.

\section{Bug Bounties in QA Strategy}

\subsection{Vulnerability Disclosure Programs vs. Bug Bounties}

A Vulnerability Disclosure Program (VDP) provides a reporting channel without monetary rewards, while a BBP actively incentivizes discovery at scale \cite{guidewire-curious}. VDPs are the baseline; BBPs drive higher scrutiny and yield more impactful findings.

\subsection{Safe Harbor}

Modern BBPs include Safe Harbor clauses protecting good-faith researchers from legal risk, addressing historical concerns under laws like the CFAA. Standards from projects like Disclose.io have improved researcher safety and organizational trust \cite{cyberscoop-safeharbor}.

\subsection{Quality Metrics}

BBPs are evaluated using metrics distinct from traditional QA:
\begin{itemize}
    \item \textbf{Signal-to-Noise Ratio} — reflects the clarity of scope and researcher understanding.
    \item \textbf{Time to Triage} and \textbf{Time to Bounty} — measure operational efficiency.
    \item \textbf{Severity Distribution} — indicates ability to attract talent capable of finding critical bugs.
    \item \textbf{MTTR} — ties findings back into DevOps agility \cite{gitkraken-dora}.
\end{itemize}

\section{Application Review and Case Studies}
\subsection{Application 1: Uber Technologies Inc.}

Uber’s bug bounty program is one of the most mature in the industry, operated through the HackerOne platform \cite{hackerone-uber}. It covers rider and driver applications, Uber Eats, Freight, and internal infrastructure, and follows a transparent “pay-for-impact’’ reward structure.

\subsubsection{Program Overview and Statistics}

Uber’s public program bases payouts primarily on impact rather than CVSS. Total rewards exceed \$4.2 million \cite{uber-update}. Typical payouts range from \$500 for low severity to \$50,000+ for critical issues.

\paragraph{Key Statistics:}
\begin{itemize}
    \item Total Bounties Paid: \(\sim\$4.2\)M
    \item Top Bounty Range: \$3{,}000–\$50{,}000
    \item Reports Resolved: \(>2{,}500\)
    \item Average Time to First Response: \(\sim 13\) hours
    \item Signal-to-Noise Ratio: \(\sim 1:6\)
\end{itemize}

Uber provides a detailed “Treasure Map’’ highlighting high-value assets such as \texttt{vault.uber.com} and \texttt{cn.uber.com} \cite{uber-policy}. From June 2025 onward, Uber will adopt a new payout table with criticals in the \$11{,}000–\$15{,}000 range \cite{uber-update}.

\subsubsection{Vulnerability Case Study: Account Takeover via API Leakage}

A major account takeover (ATO) issue was discovered by Anand Prakash in 2019, involving API information leakage and Broken Object Level Authorization (BOLA) failures \cite{apisec-uber}. An invalid \texttt{addDriver} API call leaked a victim’s UUID, enabling unauthorized access to the \texttt{getConsentScreenDetails} endpoint \cite{traceable-uber}. 

The endpoint returned personal details and the user’s authentication token (\texttt{X-Uber-Token}), enabling full account compromise \cite{digitfyi-uber}.

\paragraph{Relevance to QASST:}
The failure illustrates insufficient negative testing and inadequate backend error-message sanitization. Automated DAST tools would not detect UUID leakage without explicit rules. The bug bounty program acted as the final safety mechanism.

\subsubsection{The 2016 Breach Controversy: Governance Lessons}

In 2016, attackers obtained hardcoded AWS credentials from a private GitHub repository and stole data from 57 million users \cite{huntress-uber}. Uber paid them \$100{,}000 through its bug bounty program to conceal the breach \cite{ftc-uber}. Former CSO Joe Sullivan was later convicted of obstruction of justice for the cover-up \cite{nrf-sullivan}.

\paragraph{Relevance to QASST:}
This incident demonstrates that quality assurance requires transparency. Bug bounty programs are remediation mechanisms—not instruments for hiding breaches. Governance failures undermine systemic improvement and organizational integrity.

\input{chapters/gitlab.tex}
\subsection{Application 3: Google}
The very idea of the Google bug bounty / Vulnerability Reward program (VRP) stemmed from the approach Donald Knuth, the creator of the typing system used by this very paper you're reading  \cite{GoogleBugHunters}  \\
Back in the 70s he just kept on going for (perhaps) way too long perfecting his book "The art of computer programming", so he thought maybe it'd be a better idea to let the people submit erratas for money. Thus, a primitive form of "bug bounty" program was born. \\
Maybe try to ship fast, and allow the people to review and correct your insecure code, rewarding them for it. \\
Now, the folks over at google established a formal set of criteria by which basement-dwellers have to abide.\\

\hrule
All of the following rules and regulations can be found here: \cite{google_bughunters_invalid_reports}, \cite{google_bughunters_report} , \cite{google_bughunters_rules}, and are have specific details, that differ wether you're submitting a report for an Android, Chrome, Google or some of their open source products.
\subsubsection{Hacker Workflow ($\mathcal{W}_{\text{Hacker}}$)}
\begin{enumerate}[label=\textbf{\arabic*}., itemsep=4pt]
    \item \textbf{Prep \& Discovery:}  
    Review VRP scope/rules and look for in-scope, fixable issues.
    
    \item \textbf{Report Submission ($\mathcal{R}$):}  
    Send a clear report with repro steps, PoC, and impact.

    \item \textbf{Coordinated Disclosure:}  
    Keep it on the hush-hush until Google patches them.

    \item \textbf{Rewards:}  
    If they care, you get the moneys ($M$), possible bonuses ($B$), and the Hall of Fame credit ($H$).
\end{enumerate}

\hrule

\subsubsection{Google Workflow ($\mathcal{W}_{\text{Google}}$)}
\begin{enumerate}[label=\textbf{\arabic*}., itemsep=4pt]
    \item \textbf{Triage ($T$):}  
    Acknowledge the report, reproduce the bug, check scope and duplicates.

    \item \textbf{Severity ($S$):}  
    Rate impact and exploitability; assign Critical/High/Med/Low.

    \item \textbf{Fix ($P$):}  
    Engineering team patches the issue and deploys the fix.

    \item \textbf{Reward ($M$):}  
    VRP panel reviews severity and issues payment:  
    $M = \text{BaseReward}(S) + B$.
\end{enumerate}

\hrule

\subsubsection{VRP Lifecycle}
\[
\mathcal{R}_{\text{Submit}}
\rightarrow T
\rightarrow S
\rightarrow P
\rightarrow M
\]

\noindent


\subsubsection{An examplee noww}
There's an infamous hacker going around their forums, that seems to be Romanian: \cite{inspector-ambitious}, but he doesn't publish his reports unfortunately, so I'll instead talk about \cite{NDevTK}, who, out of all of the top spots, actually publishes their findings and replicability steps. \\

This exploit, \cite{googlesource-oauthtoken-leak, oauthtoken-leak-bughunters} has to do with the way Android source editor was parsing their \cite{android-source-editor} url's. So that, because the regex looked like:

\begin{lstlisting}[language=lua,breaklines=true]
L1.GERRIT_LINK_MATCHER =
  /(.*\/)?(.*?)\.((googlesource\.com)|(git\.corp\.google\.com))\/(.*)\/\+([a-zA-Z0-9]+)?(\/refs\/heads)?\/(.*?)[\/^](.*)/;

L1.GERRIT_LINK_MATCHER_FOR_CHANGE_FILE =
  /(.*\/)?(.*?)\.((googlesource\.com)|(git\.corp\.google\.com))\/?(\/c)?\/(.*)\/\+\/([0-9]+)\/([0-9]+)\/(.*)/;

L1.GERRIT_LINK_MATCHER_FOR_CHANGE_FILE_IN_GITLES =
  /(.*\/)?(.*?)\.((googlesource\.com)|(git\.corp\.google\.com))\/(.*)\/\+([a-zA-Z0-9]+)?(\/refs\/changes)?\/([0-9]+)\/([0-9]+)\/([0-9]+)[\/^](.*)/;
\end{lstlisting}

That meant one could try to submit edits for a URL: \url{https://android.googlesource.com/aogarantiza.com:1337#.googlesource.com/platform/build/+show/refs/heads/master/Changes.md}, but it would instead redirect to \url{aogarantiza.com:1337}, or whichever other domain / port. That attacker would be able to sniff out the Oauth token of the user's account, and could even respond back with the proper data, from Android's git repo, and the user wouldn't know a thing.


ndevtk submited the code for a whole Node web-server which is meant to run this malicious code, showcasing how they could use the Oauth token to change the user's profile description.
\begin{figure}[htbp]
    \centering
    \includegraphics[width=0.3\textwidth]{figures/bio-before.png}\\[1em]
\end{figure}

\begin{figure}[htbp]
    \centering
    \includegraphics[width=0.3\textwidth]{figures/url.png}\\[1em]
\end{figure}

\begin{figure}[htbp]
    \centering
    \includegraphics[width=0.3\textwidth]{figures/bio-now.png}
\end{figure}


\section{Bug Bounties vs. Penetration Testing}

BBPs provide continuous coverage, whereas penetration tests are time- and scope-limited \cite{bugcrowd-which}. High-impact logic bugs like Uber’s ATO example could easily fall outside the narrow time window of a traditional test. BBPs are also more cost-efficient, paying only for exploitable findings with demonstrated impact, aligning well with risk-based QA.

\section{Impact on DORA Metrics}

Bug bounty findings influence two key DORA metrics:
\begin{itemize}
    \item \textbf{Change Failure Rate} — initial increases reflect newly discovered issues; long-term reductions signal systemic improvement.
    \item \textbf{MTTR} — BBPs stress-test an organization’s ability to validate and deploy fixes quickly.
\end{itemize}

\section{Conclusion}

Bug bounty programs now function as a core component of modern software assurance, offering continuous, human-driven security validation that complements automated testing and traditional penetration assessments.


\newpage

\bibliographystyle{alpha}
\bibliography{bibliography}

\end{document}
