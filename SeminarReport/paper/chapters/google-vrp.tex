\subsection{Application 3: Google}
The very idea of the Google bug bounty / Vulnerability Reward program (VRP) stemmed from the approach Donald Knuth, the creator of the typing system used by this very paper you're reading  \cite{GoogleBugHunters}  \\
Back in the 70s he just kept on going for (perhaps) way too long perfecting his book "The art of computer programming", so he thought maybe it'd be a better idea to let the people submit erratas for money. Thus, a primitive form of "bug bounty" program was born. \\
Maybe try to ship fast, and allow the people to review and correct your insecure code, rewarding them for it. \\
Now, the folks over at google established a formal set of criteria by which basement-dwellers have to abide.\\

\hrule
All of the following rules and regulations can be found here: \cite{google_bughunters_invalid_reports}, \cite{google_bughunters_report} , \cite{google_bughunters_rules}, and are have specific details, that differ wether you're submitting a report for an Android, Chrome, Google or some of their open source products.
\subsubsection{Hacker Workflow ($\mathcal{W}_{\text{Hacker}}$)}
\begin{enumerate}[label=\textbf{\arabic*}., itemsep=4pt]
    \item \textbf{Prep \& Discovery:}  
    Review VRP scope/rules and look for in-scope, fixable issues.
    
    \item \textbf{Report Submission ($\mathcal{R}$):}  
    Send a clear report with repro steps, PoC, and impact.

    \item \textbf{Coordinated Disclosure:}  
    Keep it on the hush-hush until Google patches them.

    \item \textbf{Rewards:}  
    If they care, you get the moneys ($M$), possible bonuses ($B$), and the Hall of Fame credit ($H$).
\end{enumerate}

\hrule

\subsubsection{Google Workflow ($\mathcal{W}_{\text{Google}}$)}
\begin{enumerate}[label=\textbf{\arabic*}., itemsep=4pt]
    \item \textbf{Triage ($T$):}  
    Acknowledge the report, reproduce the bug, check scope and duplicates.

    \item \textbf{Severity ($S$):}  
    Rate impact and exploitability; assign Critical/High/Med/Low.

    \item \textbf{Fix ($P$):}  
    Engineering team patches the issue and deploys the fix.

    \item \textbf{Reward ($M$):}  
    VRP panel reviews severity and issues payment:  
    $M = \text{BaseReward}(S) + B$.
\end{enumerate}

\hrule

\subsubsection{VRP Lifecycle}
\[
\mathcal{R}_{\text{Submit}}
\rightarrow T
\rightarrow S
\rightarrow P
\rightarrow M
\]

\noindent


\subsubsection{An examplee noww}
There's an infamous hacker going around their forums, that seems to be Romanian: \cite{inspector-ambitious}, but he doesn't publish his reports unfortunately, so I'll instead talk about \cite{NDevTK}, who, out of all of the top spots, actually publishes their findings and replicability steps. \\

This exploit, \cite{googlesource-oauthtoken-leak, oauthtoken-leak-bughunters} has to do with the way Android source editor was parsing their \cite{android-source-editor} url's. So that, because the regex looked like:

\begin{lstlisting}[language=lua,breaklines=true]
L1.GERRIT_LINK_MATCHER =
  /(.*\/)?(.*?)\.((googlesource\.com)|(git\.corp\.google\.com))\/(.*)\/\+([a-zA-Z0-9]+)?(\/refs\/heads)?\/(.*?)[\/^](.*)/;

L1.GERRIT_LINK_MATCHER_FOR_CHANGE_FILE =
  /(.*\/)?(.*?)\.((googlesource\.com)|(git\.corp\.google\.com))\/?(\/c)?\/(.*)\/\+\/([0-9]+)\/([0-9]+)\/(.*)/;

L1.GERRIT_LINK_MATCHER_FOR_CHANGE_FILE_IN_GITLES =
  /(.*\/)?(.*?)\.((googlesource\.com)|(git\.corp\.google\.com))\/(.*)\/\+([a-zA-Z0-9]+)?(\/refs\/changes)?\/([0-9]+)\/([0-9]+)\/([0-9]+)[\/^](.*)/;
\end{lstlisting}

That meant one could try to submit edits for a URL: \url{https://android.googlesource.com/aogarantiza.com:1337#.googlesource.com/platform/build/+show/refs/heads/master/Changes.md}, but it would instead redirect to \url{aogarantiza.com:1337}, or whichever other domain / port. That attacker would be able to sniff out the Oauth token of the user's account, and could even respond back with the proper data, from Android's git repo, and the user wouldn't know a thing.


ndevtk submited the code for a whole Node web-server which is meant to run this malicious code, showcasing how they could use the Oauth token to change the user's profile description.
\begin{figure}[htbp]
    \centering
    \includegraphics[width=0.3\textwidth]{figures/bio-before.png}\\[1em]
\end{figure}

\begin{figure}[htbp]
    \centering
    \includegraphics[width=0.3\textwidth]{figures/url.png}\\[1em]
\end{figure}

\begin{figure}[htbp]
    \centering
    \includegraphics[width=0.3\textwidth]{figures/bio-now.png}
\end{figure}
