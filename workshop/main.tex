\documentclass[10pt, a4paper, twocolumn]{article}

% --- UNIVERSAL PREAMBLE BLOCK ---
\usepackage[a4paper, top=2cm, bottom=2cm, left=1.5cm, right=1.5cm]{geometry}
\usepackage{fontspec}

\usepackage[english, provide=*]{babel}
\babelprovide[import, onchar=ids fonts]{english}

% Set default/Latin font to Sans Serif in the main (rm) slot
\babelfont{rm}{Noto Sans}

% Load necessary packages for the report
\usepackage{booktabs}   % For professional tables
\usepackage{graphicx}   % For image placeholders
\usepackage{float}      % For figure positioning
\usepackage{titlesec}   % For section formatting
\usepackage{amsmath}    % For math symbols
\usepackage{enumitem}   % For better list control
\usepackage{listings}   % For code snippets
\usepackage{xcolor}     % For code coloring

% Optimize section spacing for the 2-column layout
\titlespacing*{\section}{0pt}{1.2ex plus .2ex minus .2ex}{.8ex plus .2ex}
\titlespacing*{\subsection}{0pt}{1ex plus .2ex minus .2ex}{.5ex plus .2ex}

% Define a simple code style
\lstset{
    basicstyle=\ttfamily\small,
    breaklines=true,
    frame=single,
    backgroundcolor=\color{gray!10},
    captionpos=b
}

\title{\textbf{Technical Analysis of Nintendo Switch Security Exploitation and Custom Firmware Architecture}}
\author{Stan Ioan-Victor, Cyber Security 240-2}

\date{\today}

\begin{document}

\maketitle

\begin{abstract}
Ever since I've started modded my Switch, I felt as if needed to make such a report, so that I'd document this marvel the hacking comunity put togather, and help me alos better understand the low-level intricaties that, honestly some of them I've taken for granted. So this report provides an exhaustive technical analysis of the modification of the "Mariko" revision of the Nintendo Switch (Tegra X1 T210B01). It details the transition from software-based exploitation to hardware-based Voltage Fault Injection (VFI). The analysis dissects the operation of HWFLY/Picofly modchips, the low-level architecture of the Hekate bootloader, and the kernel-level modifications performed by the Atmosphere Custom Firmware (CFW). Furthermore, it examines storage virtualization mechanisms (emuMMC) and concludes with a comparative engineering analysis between Switch modification and Android device rooting.
\end{abstract}

\section{Introduction}

The security architecture of the Nintendo Switch represents a significant evolution in console Digital Rights Management (DRM). Built upon the NVIDIA Tegra X1 system-on-chip (SoC), the console utilizes a sophisticated Chain of Trust designed to ensure that only code cryptographically signed by Nintendo can execute \cite{nvidiaTegra}.

However, the discovery of fundamental vulnerabilities in the silicon itself catalyzed a vibrant hardware hacking scene. The landscape is divided into two distinct eras defined by the hardware revision of the Tegra SoC.

\subsection{Erista vs. Mariko Silicon}
The lineage of the Tegra X1 chip is critical to understanding the security landscape:

\begin{itemize}
    \item \textbf{Erista (T210):} Found in launch units (2017). Contained the "Fusée Gelée" vulnerability (CVE-2018-6242) in the BootROM USB recovery stack. This allowed attackers to overflow a buffer during Recovery Mode (RCM) and gain arbitrary code execution before signature verification \cite{temkin2018fusee}. This was an unpatchable read-only memory (ROM) bug.
    \item \textbf{Mariko (T210B01):} Introduced in 2019 (v2, Lite, OLED). NVIDIA removed the susceptible code path in the BootROM, patched the USB stack, and hardened the hardware against simple voltage irregularities.
\end{itemize}

The shift to Mariko necessitated a return to hardware-based exploitation techniques, specifically targeting the physical integrity of the execution pipeline via modchips.

\section{Hardware Exploitation: VFI}

The modern "modchip" approach (HWFLY, Picofly) employs Voltage Fault Injection (VFI) to bypass the BootROM's signature verification. This technique, often referred to as "glitching," relies on inducing a transient fault in the CPU's logic circuitry \cite{retrosix}.

\subsection{Mechanism of Voltage Glitching}
Digital circuits \cite{balika011} rely on precise timing to propagate signals through logic gates. The propagation delay ($t_{pd}$) is inversely proportional to the supply voltage ($V_{dd}$).

If the voltage supplied to the CPU core ($VDD\_CPU$) is dropped significantly ("crowbarred") for a duration on the order of nanoseconds, signal propagation delays increase. If $t_{pd}$ exceeds the clock cycle duration ($t_{clk}$), the CPU may latch an incorrect value at the end of the cycle.

\begin{figure}[H]
  \centering
  \includegraphics[width=0.9\columnwidth]{installed-picofly.jpg}
  \caption{Installed Picofly Modchip (source: retrosix.wiki)}  
\end{figure}

The Following pics are taken from this paper: \cite{tegrax2-voltage-glitching} , which actually reffer to the Tegra X2, but a lot of the same things apply here as well.


\begin{figure}[H]
  \centering
  \includegraphics[width=0.9\columnwidth]{tegrax2/boot-flow.png}
  \caption{The boot flow of the Tegra X2 SoC}
\end{figure}

\begin{figure}[H]
  \centering
  \includegraphics[width=0.9\columnwidth]{tegrax2/signal-graph.png}
\end{figure}

Example glitch of 3 \textmu s length. $\verb|trigger_in|$ is the FPGA’s GPIO signal controlling the multiplexer, $\verb|MOSFET_Gate|$ is measured at the MOSFET’s gate, and $\verb|VDD_SOC|$ at the MOSFET’s drain. 


The attacker aims to corrupt the instruction that compares the calculated signature hash with the expected value, or the conditional branch instruction (e.g., `B.NE`) that jumps to the panic routine upon failure. By glitching this specific instruction, the processor "falls through" the security check, treating a failed signature verification as a success.

\subsection{Modchip Architecture: The RP2040}
Modern implementations often utilize the RP2040 microcontroller (Picofly). The modchip acts as a logic analyzer and pulse generator. It requires precise connections to the motherboard:
\begin{itemize}
    \item \textbf{eMMC Bus (CMD, CLK, DAT0):} Used to "sniff" communication between the SoC and storage. This allows the modchip to synchronize its internal timer with the deterministic execution of the BootROM.
    \item \textbf{CPU MOSFETs:} The glitch line used to momentarily short the CPU voltage rail to ground via high-speed MOSFETs on the modchip.
    \item \textbf{Reset Line:} To restart the console if the glitch fails (hangs the system).
\end{itemize}

The operation follows a strict state machine \cite{picoflyrepo}: Reset, Wait, Pattern Match, Delay, and Glitch. If successful, the BootROM accepts an invalid header and executes the payload residing on the SD card.

\section{Bootloader Engineering: Hekate}

Once code execution is achieved, the payload injected is typically \textbf{Hekate} (CTCaer Mod). It is a bare-metal operating system designed to prepare the hardware environment \cite{ctcaerhekate}.

\subsection{Minerva: DRAM Training}
One of Hekate's most engineering-intensive features is "Minerva." The Tegra X1's LPDDR4 memory requires precise calibration (training) at boot to account for physical variations in trace length and impedance.

The stock BootROM performs conservative training. Hekate injects \textbf{Minerva}, a custom training cell, into the BPMP (Boot and Power Management Processor). Minerva retrains the DRAM with aggressive parameters, achieving higher bandwidth (up to 1600MHz+ compared to stock 1333MHz) and lower latency.

\subsection{Payload Injection}
Hekate parses `hekate\_ipl.ini` to determine boot targets. It acts similarly to GRUB on Linux, offering a menu to chainload other payloads. The primary target for most users is `fusee.bin`, which bootstraps the Atmosphere CFW.

\section{Custom Firmware: Atmosphere}

Atmosphere is not a permanent firmware replacement but a set of runtime patches. It is a masterpiece of reverse engineering, consisting of userland system modules and a complete kernel reimplementation \cite{sciresmatmosphere}.

\subsection{Mesosphere: The Custom Kernel}
Originally, Atmosphere patched the official Nintendo kernel. However, later versions introduced \textbf{Mesosphere}, a complete open-source reimplementation of the Switch Horizon kernel.
\begin{itemize}
    \item \textbf{Accuracy:} It replicates the behavior of the official kernel 1:1 to ensure game compatibility.
    \item \textbf{Instrumentation:} It allows for extensive debugging and hooking that would be impossible with binary patches alone.
    \item \textbf{Optimization:} It fixes bugs present in the official Nintendo kernel.
\end{itemize}

\begin{figure}[H]
  \centering
  \includegraphics[width=0.9\columnwidth]{system-architecture.png}
  \caption{Kernel et. al. Layering and architecture \cite{rousseltarbouriech2019methodicallydefeatingnintendoswitch}}
\end{figure}

\subsection{Exosphère (Secure Monitor)}
Running at Exception Level 3 (EL3), Exosphère reimplements the TrustZone Secure Monitor. It handles Secure Monitor Calls (SMCs) and manages hardware security keys.

Crucially, Exosphère enables \textbf{Prodinfo Blanking}. It intercepts reads to the calibration partition where the console's unique Serial Number and Certificates are stored, effectively creating a "stealth mode" to avoid account bans.

\subsection{Stratosphère (Userland Modules)}
This layer consists of custom system modules that intercept Inter-Process Communication (IPC) calls.

\subsubsection{fs\_mitm}
The "File System Man-in-the-Middle" module hooks the file system service (`fsp-srv`). When a game requests a file, `fs\_mitm` checks a specific directory on the SD card (LayeredFS). This allows users to replace game assets transparently without breaking the digital signature of the game container itself \cite{switchbrew}.

\subsubsection{loader}
The `loader` system module is patched to bypass checks that prevent unsigned NRO (Nintendo Relocatable Object) executables from running and to implement "Title Override."

\section{Storage Virtualization: emuMMC}

emuMMC (Emulated Multi-Media Card) decouples the operating system from the physical NAND flash.

\subsection{Block-Level Redirection}
The Switch OS communicates with storage via the MMC driver. emuMMC patches this driver. When the OS attempts to read/write to Device ID 0 (Internal NAND), the driver redirects these Logical Block Addresses (LBAs) to an offset on the microSD card (Device ID 1).

\begin{table}[htbp]
\centering
\caption{emuMMC Implementation Types}
\label{tab:emummc}
\small
\begin{tabular}{@{}lp{2.5cm}p{2.5cm}@{}}
\toprule
\textbf{Metric} & \textbf{Partition-Based} & \textbf{File-Based} \\ \midrule
Structure & Raw hex data at SD offset & Split files (00, 01...) \\
I/O Path & Direct sector access & Through FAT32 \\
Speed & High (Near Native) & Moderate \\
\bottomrule
\end{tabular}
\end{table}

\section{Comparisons}

\subsection{Applet vs. Application Mode}
Execution of unsigned code faces memory constraints \cite{switchbrew}:
\begin{itemize}
    \item \textbf{Applet Mode:} Launched via Album. Restricted to $\approx$448 MB RAM.
    \item \textbf{Application Mode:} Launched by holding `R` over a game. Grants full access to $\approx$3.2 GB RAM and GPU resources.
\end{itemize}

\subsection{Android vs. Switch Security}
\textbf{Android (Rooting):} Relies on the Unix UID model (getting UID 0). It is typically persistent (flashed to `boot.img`).

\textbf{Switch (Atmosphere):} Horizon is a microkernel using Capability-based Security (Handles). There is no "Root User." Modding requires patching specific capability checks (Service Management). It is volatile (RAM-only); patches must be reapplied at every cold boot via the VFI exploit chain.

\section{Conclusion}

The modification of a "Mariko" Switch is a complex interplay of hardware physics and software engineering. It begins with VFI to bypass the hardware Root of Trust, followed by the initialization of Hekate and the runtime patching of the microkernel by Atmosphere.

% References
\bibliographystyle{ieeetr}
\bibliography{sample}

\end{document}
