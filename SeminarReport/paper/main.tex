\documentclass{article}

% Language setting
% Replace `english' with e.g. `spanish' to change the document language
\usepackage[english]{babel}

% Set page size and margins
% Replace `letterpaper' with `a4paper' for UK/EU standard size
\usepackage[letterpaper,top=2cm,bottom=2cm,left=3cm,right=3cm,marginparwidth=1.75cm]{geometry}

% Useful packages
\usepackage{amsmath}
\usepackage{graphicx}
\usepackage[colorlinks=true, allcolors=blue]{hyperref}
\usepackage{xcolor}
\usepackage{amssymb}
\usepackage{fancyhdr}
\usepackage{enumitem}
\usepackage[utf8]{inputenc}
\usepackage{booktabs} % For better looking tables
\usepackage{caption} % For table caption control

\renewcommand{\headrulewidth}{0.4pt}
\renewcommand{\footrulewidth}{0pt}

\title{\textbf{SEMINAR REPORT TITLE}\\
Seminar Report}

\author{TEAM:\\
Stan Ioan-Victor, ioan.victor.stan@ubbcluj.ro\\
Sebastian-George Hojda, sebastian.hojda@ubbcluj.ro\\
Timotei Copaciu, timotei.copaciu@ubbcluj.ro
}


\pagestyle{fancy}

\date{\today}


\begin{document}

\maketitle

\tableofcontents

\newpage

\section{Introduction}
\label{label:Introduction}

\textcolor{blue}{Include a paragraph with details about the topic approached. Discuss in another 1-2 paragraphs the relevance of the topic within the security domain.
}


\section{Related Concepts / Previous Work}
\label{label:RelatedConcepts}

\textcolor{blue}{\textit{For the practical approach} $->$ Introduce the most important concepts related to the addressed topic. Include references where required and cite them in this way \cite{Vuln001}.\\
}


\section{\textit{Topic-based Sections}}
\label{label:TopicBasedSection}

\textcolor{blue}{\textit{For the practical approach} $->$ Include \textit{in extenso} details about the targeted topic. If needed, use additional sections and subsections. \\}

\subsection{Application 3: Google}
The very idea of the Google bug bounty / Vulnerability Reward program (VRP) stemmed from the approach Donald Knuth, the creator of the typing system used by this very paper you're reading  \cite{GoogleBugHunters}  \\
Back in the 70s he just kept on going for (perhaps) way too long perfecting his book "The art of computer programming", so he thought maybe it'd be a better idea to let the people submit erratas for money. Thus, a primitive form of "bug bounty" program was born. \\
Maybe try to ship fast, and allow the people to review and correct your insecure code, rewarding them for it. \\
Now, the folks over at google established a formal set of criteria by which basement-dwellers have to abide.\\

\hrule
All of the following rules and regulations can be found here: \cite{google_bughunters_invalid_reports}, \cite{google_bughunters_report} , \cite{google_bughunters_rules}, and are have specific details, that differ wether you're submitting a report for an Android, Chrome, Google or some of their open source products.
\subsubsection{Hacker Workflow ($\mathcal{W}_{\text{Hacker}}$)}
\begin{enumerate}[label=\textbf{\arabic*}., itemsep=4pt]
    \item \textbf{Prep \& Discovery:}  
    Review VRP scope/rules and look for in-scope, fixable issues.
    
    \item \textbf{Report Submission ($\mathcal{R}$):}  
    Send a clear report with repro steps, PoC, and impact.

    \item \textbf{Coordinated Disclosure:}  
    Keep it on the hush-hush until Google patches them.

    \item \textbf{Rewards:}  
    If they care, you get the moneys ($M$), possible bonuses ($B$), and the Hall of Fame credit ($H$).
\end{enumerate}

\hrule

\subsubsection{Google Workflow ($\mathcal{W}_{\text{Google}}$)}
\begin{enumerate}[label=\textbf{\arabic*}., itemsep=4pt]
    \item \textbf{Triage ($T$):}  
    Acknowledge the report, reproduce the bug, check scope and duplicates.

    \item \textbf{Severity ($S$):}  
    Rate impact and exploitability; assign Critical/High/Med/Low.

    \item \textbf{Fix ($P$):}  
    Engineering team patches the issue and deploys the fix.

    \item \textbf{Reward ($M$):}  
    VRP panel reviews severity and issues payment:  
    $M = \text{BaseReward}(S) + B$.
\end{enumerate}

\hrule

\subsubsection{VRP Lifecycle}
\[
\mathcal{R}_{\text{Submit}}
\rightarrow T
\rightarrow S
\rightarrow P
\rightarrow M
\]

\noindent


\subsubsection{An examplee noww}
There's an infamous hacker going around their forums, that seems to be Romanian: \cite{inspector-ambitious}, but he doesn't publish his reports unfortunately, so I'll instead talk about \cite{NDevTK}, who, out of all of the top spots, actually publishes their findings and replicability steps. \\

This exploit, \cite{googlesource-oauthtoken-leak, oauthtoken-leak-bughunters} has to do with the way Android source editor was parsing their \cite{android-source-editor} url's. So that, because the regex looked like:

\begin{lstlisting}[language=lua,breaklines=true]
L1.GERRIT_LINK_MATCHER =
  /(.*\/)?(.*?)\.((googlesource\.com)|(git\.corp\.google\.com))\/(.*)\/\+([a-zA-Z0-9]+)?(\/refs\/heads)?\/(.*?)[\/^](.*)/;

L1.GERRIT_LINK_MATCHER_FOR_CHANGE_FILE =
  /(.*\/)?(.*?)\.((googlesource\.com)|(git\.corp\.google\.com))\/?(\/c)?\/(.*)\/\+\/([0-9]+)\/([0-9]+)\/(.*)/;

L1.GERRIT_LINK_MATCHER_FOR_CHANGE_FILE_IN_GITLES =
  /(.*\/)?(.*?)\.((googlesource\.com)|(git\.corp\.google\.com))\/(.*)\/\+([a-zA-Z0-9]+)?(\/refs\/changes)?\/([0-9]+)\/([0-9]+)\/([0-9]+)[\/^](.*)/;
\end{lstlisting}

That meant one could try to submit edits for a URL: \url{https://android.googlesource.com/aogarantiza.com:1337#.googlesource.com/platform/build/+show/refs/heads/master/Changes.md}, but it would instead redirect to \url{aogarantiza.com:1337}, or whichever other domain / port. That attacker would be able to sniff out the Oauth token of the user's account, and could even respond back with the proper data, from Android's git repo, and the user wouldn't know a thing.


ndevtk submited the code for a whole Node web-server which is meant to run this malicious code, showcasing how they could use the Oauth token to change the user's profile description.
\begin{figure}[htbp]
    \centering
    \includegraphics[width=0.3\textwidth]{figures/bio-before.png}\\[1em]
\end{figure}

\begin{figure}[htbp]
    \centering
    \includegraphics[width=0.3\textwidth]{figures/url.png}\\[1em]
\end{figure}

\begin{figure}[htbp]
    \centering
    \includegraphics[width=0.3\textwidth]{figures/bio-now.png}
\end{figure}

\input{chapters/gitlab.tex}

\subsection{Examples / Results}

\textcolor{blue}{Where needed, please include examples and experiment results using tables (Table \ref{tab:TCs1}) and screenshots or images (Figure \ref {fig:frog}) and cite them. }

\begin{table} [htpb]
\centering
\begin{tabular}{l|l|l|l|l}
Feature & TC ID & Input1 & Input2 & Expected Output \\ \hline
F001  &TC01 & 42 & 15 & 100\\
F001  &TC02 & 1 & -2 & 3\\
F002  &TC03 & 111 & 90 & -74
\end{tabular}
\caption{\label{tab:TCs1}TCs table.}
\end{table}

\begin{figure} [htpb]
\centering
\includegraphics[width=0.25\linewidth]{Figures/frog.jpg}
\caption{\label{fig:frog}This frog was uploaded via the file-tree menu.}
\end{figure}


\section{Conclusions}
\label{label:Conclusions}

\textcolor{blue}{Include conclusions about the investigated topic, lessons learned, and personal considerations while working as a team on the selected topic, summarized in 3-4 paragraphs. When listing the contributions of each team member, you may have something like this:}

\textcolor{blue}{\begin{itemize}
\vspace{-0.5cm}
    \item 1st student name - sections \ref{label:Introduction} and \ref{label:RelatedConcepts}
    \item 2nd student name - sections \ref{label:RelatedConcepts}, \ref{label:TopicBasedSection}, and \ref{label:Conclusions}
    \item etc.
\end{itemize}}

\section{Other sections...}

\textcolor{blue}{Please remove this section and all subsections.}
\subsection{How to include Figures}

First you have to upload the image file from your computer using the upload link in the file-tree menu. Then use the includegraphics command to include it in your document. Use the figure environment and the caption command to add a number and a caption to your figure. See the code for Figure \ref{fig:frog} in this section for an example.

Note that your figure will automatically be placed in the most appropriate place for it, given the surrounding text and taking into account other figures or tables that may be close by. You can find out more about adding images to your documents in this help article on \href{https://www.overleaf.com/learn/how-to/Including_images_on_Overleaf}{including images on Overleaf}.


\subsection{How to add Citations and a References List}

You can simply upload a \verb|.bib| file containing your BibTeX entries, created with a tool such as JabRef. You can then cite entries from it, like this: \cite{greenwade93}. Just remember to specify a bibliography style, as well as the filename of the \verb|.bib|. You can find a \href{https://www.overleaf.com/help/97-how-to-include-a-bibliography-using-bibtex}{video tutorial here} to learn more about BibTeX.

If you have an \href{https://www.overleaf.com/user/subscription/plans}{upgraded account}, you can also import your Mendeley or Zotero library directly as a \verb|.bib| file, via the upload menu in the file-tree.

\newpage

\bibliographystyle{alpha}
\bibliography{sample}

\end{document}
