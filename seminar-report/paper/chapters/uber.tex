\subsection{Application 1: Uber Technologies Inc.}

This section provides an in-depth review of the bug bounty ecosystems of Uber, Google and GitLab. For each application, we analyze the program structure, operational statistics, and dissect specific, high-impact vulnerabilities that exemplify the critical role these programs play in assuring software quality.

Uber’s bug bounty program is one of the most mature in the industry, operated through the HackerOne platform \cite{hackerone-uber}. It covers rider and driver applications, Uber Eats, Freight, and internal infrastructure, and follows a transparent “pay-for-impact’’ reward structure.

\subsubsection{Program Overview and Statistics}

Uber’s public program bases payouts primarily on impact rather than CVSS. Total rewards exceed \$4.2 million \cite{uber-update}. Typical payouts range from \$500 for low severity to \$50,000+ for critical issues.

\paragraph{Key Statistics:}
\begin{itemize}
    \item Total Bounties Paid: \(\sim\$4.2\)M
    \item Top Bounty Range: \$3{,}000 to \$50{,}000
    \item Reports Resolved: \(>2{,}500\)
    \item Average Time to First Response: \(\sim 13\) hours
    \item Signal-to-Noise Ratio: \(\sim 1:6\)
\end{itemize}

Uber provides a detailed “Treasure Map’’ highlighting high-value assets such as \texttt{vault.uber.com} and \texttt{cn.uber.com} \cite{uber-policy}. From June 2025 onward, Uber will adopt a new payout table with criticals in the \$11{,}000 to \$15{,}000 range \cite{uber-update}.


\begin{figure}[htbp]
    \centering
    \includegraphics[width=1\linewidth]{figures/uber01.png}
    \caption{HackerOne's Uber Bug Bounty Program Page \cite{hackerone-uber}}.
\label{fig:gitlab-example}
\end{figure}


\subsubsection{Vulnerability Case Study: Account Takeover via API Leakage}

A major account takeover (ATO) issue was discovered by Anand Prakash in 2019, involving API information leakage and Broken Object Level Authorization (BOLA) failures \cite{apisec-uber}. An invalid \texttt{addDriver} API call leaked a victim’s UUID, enabling unauthorized access to the \texttt{getConsentScreenDetails} endpoint \cite{traceable-uber}. 

\begin{figure}[htbp]
    \centering
    \includegraphics[width=0.8\linewidth]{figures/uber02.png}
    \caption{getConsentScreenDetails endpoint request\cite{traceable-uber}}.
\label{fig:gitlab-example}
\end{figure}

The endpoint returned personal details and the user’s authentication token (\texttt{X-Uber-Token}), enabling full account compromise \cite{digitfyi-uber}.

\begin{figure}[htbp]
    \centering
    \includegraphics[width=0.5\linewidth]{figures/uber03.png}
    \caption{getConsentScreenDetails endpoint response payload\cite{traceable-uber}}.
\label{fig:gitlab-example}
\end{figure}

\paragraph{Relevance to QASST:}
The failure illustrates insufficient negative testing and inadequate backend error-message sanitization. Automated DAST tools would not detect UUID leakage without explicit rules. The bug bounty program acted as the final safety mechanism.

\subsubsection{The 2016 Breach Controversy: Governance Lessons}

In 2016, attackers obtained hardcoded AWS credentials from a private GitHub repository and stole data from 57 million users \cite{huntress-uber}. Uber paid them \$100{,}000 through its bug bounty program to conceal the breach \cite{ftc-uber}. Former CSO Joe Sullivan was later convicted of obstruction of justice for the cover-up \cite{nrf-sullivan}.

\paragraph{Relevance to QASST:}
This incident demonstrates that quality assurance requires transparency. Bug bounty programs are remediation mechanisms—not instruments for hiding breaches. Governance failures undermine systemic improvement and organizational integrity.
